\chapter{Введение в Upstart} \label{sec:Introduction}

\section{Что такое Upstart?}

Цитата с http://upstart.ubuntu.com: "Upstart --- это событийно-ориентированная замена традиционному демону /sbin/init, запускающая задачи и демоны во время загрузки системы и останавливающая их в момент ее выключения, а так же наблюдающая за ними во время работы системы".

Стартовый процесс, он же - система инициализации в Unix и Linux представляет собой процесс с PID равным 1. Можно сказать, что это процесс, который стартует первым с момента запуска системы, исключая запуск ядра и/или initramfs. Как видно из приведенной цитаты, Upstart является заменой традиционному для основанных на архитектуре System V Release 4 систем.

\subsection{Надежность}

Upstart написана с использованием служебной библиотеки NIH ("libnih"). Это очень маленькая, эффективная и безопасная библиотека, содержащая базовые функции. Она разработана для приложений, рано стартующих при загрузке системы. Надежность и безопасность являются абсолютно необходимыми качествами системы инициализации по следующим причинам:
\begin{itemize}
\item она выполняется с правами root.
\item она ответственна за управление важнейшими системными службами.
\item если процесс инициализации обрывается по любой причине, следует паника ядра.
\end{itemize}
Чтобы обеспечить надежность и избежать регрессий, и Upstart и библиотека NIH поставляются с послым набором юнит-тестов. Подробнее смотрите в соответствующей главе.
\subsection{История возникновения}
Upstart был создан из-за неустранимых ограничений существовавших в то время систем инициализаций. Эти системы могут быть отнесены к двум типам: \begin{itemize}
\item Системы инициализации в стиле System V.
\item Системы инициализации основанные на зависимостях.
\end{itemize}
Для лучшего понимания почему Upstart был написан и почему его дизайн был революционным следует рассмотреть эти виды систем инициализаций.
\subsubsection{Upstart и SysV}
\paragraph{Преимущества SysV}
\subparagraph{Простота}
Создание файлов сервисов в системе инициализации SysV является очень простой задачей, так как они представляют собой обычные shell-скрипты. Для включения или исключения демона в заданном уровне запуска достаточно всего лишь создать символическую ссылку на сценарий его выполнения в заданной директории или множестве директорий.
\subparagraph{Гарантированный порядок запуска и остановки демонов}
Это достигается за счет последовательного исполнения скриптов, на которые указывают соответствующие символические ссылки. Относительный порядок, в котором исполняются эти сценарии определяется их именованием: те скрипты, которые стоят в лексикографическом порядке первыми исполняются раньше тех скриптов, которые идут после них.
\paragraph{Ограничения SysV}
\subparagraph{Неоптимальная производительность}
Традиционная система последовательной загрузки и соответствующее ей время запуска сервисов были вполне приемлемы на момент своего изобретения, однако на сегодняшний день она кажется "тормозной" так как не использует никаких особенностей оборудования и не поддерживает параллельность запуска или остановки задач.

SysV init была простой и эффективной для управления системными администраторами.Тем не менее эта модель не до конца использует современные технические возможности, особенно в области распараллеливания запуска демонов.

Наиболее часто встречающийся путь обхода строго последовательного запуска служб состоял в фоновом выполнении своих демонов, что позволяло добиться некоторой степени параллелизма. Повсеместная распространенность этого хака ясно указывала на концептуальный недостаток системы инициализации в стиле SysV.  
\subparagraph{Ориентированность на серверы}
В дни традиционных Unix-систем с их мэйнфреймами и сотнями рабочих терминалов, когда перезагрузки были редки, подход SysV был оправданным. Если аппаратура нуждалась в замене, то главная станция выключалась, аппаратура заменялась и станция включалась вновь. 

Однако с тех пор мир изменился, с точки зрения команды разработчиков Ubuntu нынешние пользователи могут захотеть перезагрузить свою рабочую систему когда угодно. \footnote{Не забываем также о появлении большого количества "живых" систем, где сеансы работы носят эпизодический характер. --- прим. переводчика.}
\subparagraph{Состав оборудования предполагается неизменным}\footnote{На время сеанса работы с системой --- прим. переводчика}
Современные Linux -- системы часто имеют дело с оборудованием, поддерживающим "горячие" замены и должны адекватно отражать подобные вещи в своей работе. Традиционный подход, основанный на статических скриптах инициализации, не предоставляет такой возможности.
\subparagraph{Каждый сценарий делает ненужную работу} 
Большинство скриптов инициализации являются шаблонными. Например, они выполняют такие первоначальные проверки как \begin{itemize}
\item Единственность экземпляра запущенного демона.

\end{itemize}
\subsubsection{Upstart и системы, ориентированные на зависимости}