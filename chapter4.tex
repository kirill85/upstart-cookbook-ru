\chapter{Конфигурация сервисов Upstart}
Эта глава содержит в себе описания всех инструкций или станс сервисов Upstart. В таблице ниже приводится их список, разбитый на категории.
\begin{table}
\begin{tabular}{|c|c|c|}
\hline Категория & Станса & В какой версии появилась \\ 
\hline \multirow{6}{*}{Работа с процессами} & \stanza{exec} &  \\
 & \stanza{pre--start} & \\
 & \stanza{post--start} & \\
 & \stanza{pre--stop} & \\
 & \stanza{post--stop} & \\
 & \stanza{script} & \\ 
\hline \multirow{3}{*}{Условия на события} & \stanza{manual} & 0.67 \\
 & \stanza{start on} & \\
 & \stanza{stop on} & \\
\hline \multirow{2}{*}{Окружение сервиса} & \stanza{env} & \\
& \stanza{export} & \\
\hline \multirow{4}{*}{Сервисы, задачи и их перезапуск} & \stanza{normal exit} & \\
& \stanza{respawn} & \\
& \stanza{respawn limit} & \\
& \stanza{task} & \\
\hline Работа с экземплярами сервиса & \stanza{instance} & \\
\hline \multirow{5}{*}{Документирование сервиса} & \stanza{author} & \\
& \stanza{description} & \\
& \stanza{emits} & \\
& \stanza{version} & \\
& \stanza{usage} & 1.5 \\
\hline \multirow{14}{*}{Работа с окружением процесса} & \stanza{apparmor load} & 1.9 \\
& \stanza{apparmor switch} & 1.9 \\
& \stanza{console none} & \\
& \stanza{console log} & 1.4 \\
& \stanza{console output} & \\
& \stanza{console owner} & \\
& \stanza{chdir} & \\
& \stanza{limit} & \\
& \stanza{nice} & \\
& \stanza{oom score} & \\
& \stanza{setgid} & 1.4 \\
& \stanza{setuid} & 1.4 \\
& \stanza{umask} & \\
\hline \multirow{6}{*}{Управление процессом} & \stanza{expect fork} & \\
& \stanza{expect daemon} & \\
& \stanza{expect stop} & \\
& \stanza{kill signal} & 1.3 \\
& \stanza{kill timeout} & \\
& \stanza{reload signal} & 1.10 \\
\hline
\end{tabular}  
\end{table}
\newpage
Далее указаны значения станс и приведены примеры кода для каждой из них.
\subsection{Apparmor}
\paragraph{apparmor load} --- загружает указанный профиль доступа AppArmor для данного сервиса в ядро перед тем как запускать этот сервис. Дальнейшее выполнение его главного процесса продолжится уже под управлением загруженного профиля. Синтаксис: \stanza{apparmor load}{<profile-path>}. Пример: \begin{alltt}
\stanza{apparmor load}{\filepath{etc/apparmor.d/usr.sbin.cupsd}}
\stanza{exec}{\filepath{usr/sbin/cupsd -F}}
\end{alltt}

Замечания: \begin{itemize}
\item file-path должен представлять собой полный путь к профилю AppArmor.
\item Запуск сервиса завершится неудачей если указанного профиля не существует, либо же он не смог загрузиться.
\end{itemize}