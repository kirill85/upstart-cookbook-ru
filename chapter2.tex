\chapter{Понятия и терминология}
Главными понятиями у Upstart являются события и задания ("jobs"). Важно понимать различия между ними.
\section{Задания}
"Единицей работы" в Upstart является задача ("task") или служба ("service"). Задание, в не зависимости от его типа, описывается в файле конфигурации задания.
\subsection{Виды заданий}
\paragraph{Разовое задание ("task")}

Разовое задание представляет собой короткоживущий процесс, то есть имеющий конечное время на выполнение и безусловно завершающийся.

Например, удаление некоторого файла представляет собой такое задание: есть старт программы, сама операция его удаления --- возможно занимающая время, если файл большой --- и завершение задачи по окончании удаления файла.

В дальнейшем такие демоны будут именоваться задачами ("task job").

\paragraph{Сервис ("service job")}

Сервис или демон представляет собой долгоживущий процесс --- возможно, вплоть до останова системы. 
В противоположность заданию, сервис никогда не завершает свою работу сам.

Примерами сервисов могут служить базы данных, веб-серверы и тому подобное.
\paragraph{Абстрактный сервис}

Это еще один тип сервиса, отличающийся от первых двух тем, что в его файле конфигурации 
отсутствует ключевое слово exec или секция script. Каждый такой сервис может быть запущен или остановлен, но он не порождает дочернего процесса (PID). Фактически такие сервисы "выполняются" до тех пор пока явно не останавливаются администратором системы. Абстрактные сервисы существуют в Upstart  только по причине их чрезвычайной полезности. Например такие сервисы могут служить цели синхронизации.
\subsection{Состояния сервисов}

В таблице ниже приводятся возможные состояния, в которых могут находиться сервисы и допустимые переходы между ними.
\begin{table}[h]
\begin{tabular}{|l|l|l|}
\hline Текущее состояние &  Команда  \\ 
\hline  & start & stop \\ 
\hline waiting & starting & не определено \\ 
\hline starting & pre-start & stopping \\ 
\hline pre-start & spawned & stopping \\ 
\hline spawned & post-start & stopping \\ 
\hline post-start & running & stopping \\
\hline running & stopping & pre-stop или stopping \footnote{} \\
\hline pre-stop & running & stopping \\
\hline stopping & killed & killed \\
\hline killed & post-stop & post-stop \\
\hline post-stop & starting & waiting \\
\hline
\end{tabular} 
\end{table}
Например, если сервис находится в состоянии starting (стартует) и ему была отдана команда start, то он перейдет в состояние pre-start (приготовиться к запуску).

Заметьте, что эти переходы могут быть сделаны так быстро, что вы не успеете заметить этого в выводе утилиты initctl. Изменить эту ситуацию можно командой initctl log-priority с параметром равным info
или debug.

Пояснения к обозначениям состояний: \begin{itemize}
\item waiting --- начальное состояние.
\item starting --- сервис начинает запускаться.
\item pre-start --- выполняется секция pre-start в файле конфигурации сервиса.
\item spawned --- исполняется инструкция exec или секция script.
\item post-start --- исполняются инструкции в секции post-start файла конфигурации.
\item running --- состояние, следующее за post-start, означающее, что сервис исполняется (при этом ему необязательно соответствует PID).
\item pre-stop --- выполняется секция pre-stop.
\item stopping --- следует за pre-stop. 
\item killed --- сервис начинает завершаться.
\item post-stop --- завершающие действия по очистке окружения и возврату ресурсов.
\end{itemize}
\paragraph{Просмотр изменений состояний сервиса}
Подробно увидеть переходы между состояниями сервиса можно увидеть одним из трех способов: \begin{enumerate}
\item Выставлением параметра log-priority в значение debug.
\item С помощью утилиты tail с аргументами -f <service.log>
\item Командами start/restart/stop или отправкой соответствующих событий.
\end{enumerate}
\section{Окружение сервиса}
Пока Upstart исполняет сервис, он предоставляет ему очень ограниченное системное окружение, 
содержащее две основные переменные: TERM и PATH. Сам Upstart также устанавливает специальные переменные, которые может использовать процесс. Подробнее об этом будет изложено ниже. Если ваши системные сервисы 
требуют введения специальных переменных, то вы можете использовать для этого ключевые слова env и export.

Сервисы для сессий устроены по-другому: для них также доступно определение переменных с помощью env и export, но они также наследуют переменные окружения из своей сессии.
\section{Файл конфигурации сервиса}
Сервис определяется в файлах конфигурации сервиса, или проще говоря в .conf файле. Он представляет собой простой текстовый файл, содержащий одну или более стансу, или ключевую инструкцию, Файлы конфигурации именуются по правилу <name>.conf, где  name --- имя приложения или сервиса. .conf файлы бывают двух типов: 
системные и пользовательские. В дальнейшем под словом "сервис" будет пониматься его файл конфигурации.
\paragraph{Системные сервисы}
Файлы системных сервисов по умолчанию располагаются в директории /etc/init. Однако это умолчание можно изменить с помощью параметра --confdir=<dir> для демона /sbin/init. 
\paragraph{Пользовательский сервис}
Устарел, заменен в версии 1.7 сервисом сессии.

Upstart 1.3 ввел понятие пользовательских сервисов, выполняющихся от имени непривилегированного пользователя. По умолчанию файлы конфигурации таких сервисов располагаются в директории \textdollar HOME/.init/. Сейчас эта возможность более недоступна, начиная с версии Ubuntu 11.10 и выше.
Синтаксис таких файлов полностью одинаков синтаксису файлов системных сервисов.\footnote{Недопустимо, однако, совпадение имен пользовательского и системного сервиса.}
\paragraph{Сервис сессии}
Начиная с версии Upstart 1.7, были введены сервисы в рамках сессии, во всем подобные пользовательским сервисам за исключением того, что они выполняются с PID унаследованным от текущей сессии, а не 1 как
раньше. 
Файлы их конфигурации могут располагаться в следующих директориях: \begin{itemize}
\item \textdollar XDG\_CONFIG\_HOME/upstart/ (или \textdollar HOME/.config/upstart).
\item \textdollar HOME/.init/.
\item \textdollar XDG\_CONFIG\_DIRS.
\item /usr/share/upstart/sessions.
\end{itemize}
Имя каждого задания принимается базовому имени его файла конфигурации за вычетом имен всех вышележащих директорий. Например, если файл конфигурации расположен в \textdollar HOME/.config/upstart/hello/world.conf, то имя сервиса будет "hello/world".