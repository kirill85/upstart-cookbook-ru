\chapter{Понятия и терминология}
Главными понятиями у Upstart являются события и задания ("jobs"). Важно понимать различия между ними.
\section{Задания}
"Единицей работы" в Upstart является задача ("task") или служба ("service"). Задание, в не зависимости от его типа, описывается в файле конфигурации задания.
\subsection{Виды заданий}
\paragraph{Разовое задание ("task")}
Разовое задание представляет собой короткоживущий процесс, то есть имеющий конечное время на выполнение и безусловно завершающийся.

Например, удаление некоторого файла представляет собой такое задание: есть старт программы, сама операция его удаления --- возможно занимающая время, если файл большой --- и завершение задачи по окончании удаления файла.

В дальнейшем такие демоны будут именоваться задачами ("task job").

\paragraph{Сервис ("service job")}
Сервис или демон представляет собой долгоживущий процесс --- возможно, вплоть до останова системы. 
В противоположность заданию, сервис никогда не завершает свою работу сам.

Примерами сервисов могут служить базы данных, веб-серверы и тому подобное.